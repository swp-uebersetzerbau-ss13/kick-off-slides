\documentclass[ucs,9pt]{beamer}

% Copyright 2004 by Till Tantau <tantau@users.sourceforge.net>.
%
% In principle, this file can be redistributed and/or modified under
% the terms of the GNU Public License, version 2.
%
% However, this file is supposed to be a template to be modified
% for your own needs. For this reason, if you use this file as a
% template and not specifically distribute it as part of a another
% package/program, I grant the extra permission to freely copy and
% modify this file as you see fit and even to delete this copyright
% notice.
%
% Modified by Tobias G. Pfeiffer <tobias.pfeiffer@math.fu-berlin.de>
% to show usage of some features specific to the FU Berlin template.

% remove this line and the "ucs" option to the documentclass when your editor is not utf8-capable
\usepackage[utf8x]{inputenc}    % to make utf-8 input possible
\usepackage[english,german]{babel}     % hyphenation etc., alternatively use 'german' as parameter
\usepackage{hyperref}

\include{fu-beamer-template}  % THIS is the line that includes the FU template!

\usepackage{arev,t1enc} % looks nicer than the standard sans-serif font
% if you experience problems, comment out the line above and change
% the documentclass option "9pt" to "10pt"

% image to be shown on the title page (without file extension, should be pdf or png)
\titleimage{fu_500}

\newcommand{\email}[1]{\href{mailto:#1}{#1}}

\title{SWP Übersetzerbau im SS 13}
\subtitle{Einführung und Organisatorisches}
\author{Till~Zoppke \and Maximilian~Konzack \and Yves~Müller}
\institute[FU Berlin]{Freie Universität Berlin}

\date{Auftaktveranstaltung am 13.~April 2013}


% you can redefine the text shown in the footline. use a combination of
% \insertshortauthor, \insertshortinstitute, \insertshorttitle, \insertshortdate, ...
\renewcommand{\footlinetext}{\insertshortinstitute, \insertshorttitle, \insertshortdate}

% Delete this, if you do not want the table of contents to pop up at
% the beginning of each subsection:
%\AtBeginSubsection[]
%{
%  \begin{frame}<beamer>{Outline}
%    \tableofcontents[currentsection,currentsubsection]
%  \end{frame}
%}

\begin{document}

\begin{frame}[plain]
  \titlepage
\end{frame}

\begin{frame}{Übersicht}
  \tableofcontents
\end{frame}

\section{Projektidee}
\begin{frame}
  \frametitle{Entwicklung eines Compilers im Team}
  \begin{center}
    \includegraphics[width=0.7\textwidth]{pair_prog}
  \end{center}
\end{frame}
\begin{frame}
  \frametitle{Modularer Compiler}
  \begin{block}{Idee}
    \begin{itemize}
      \item Implementierung eines Übersetzers
      \item soll im Rahmen der Übersetzerbau genutzt werden können
      \item Quellsprache ist imperativ und statisch typisiert
    \end{itemize}
  \end{block}

  \begin{columns}
    \begin{column}{0.3\textwidth}
  \begin{center}
    \includegraphics[height=3cm]{java_duke}
  \end{center}
\end{column}
    \begin{column}{0.7\textwidth}
  \begin{block}{Ziele}
      \begin{description}
          \item[modular] Abgrenzung gegenüber anderen Modulen
          \item[einfach] Verwendbarkeit mit anderen Komponenten
          \item[getestet] Black- und Whiteboxtests
          \item[dokumentiert] im Quellcode und als Text
      \end{description}
  \end{block}
\end{column}
  \end{columns}
\end{frame}

\begin{frame}
    \frametitle{Veranschaulichung der zu erstellenden Artefakte}
    \begin{center}
        \includegraphics[width=0.9\textwidth]{pipeline}
    \end{center}
    \begin{itemize}
        \item bekannte Aufteilung in Front- und Backend
        \item Projektmanagement bezogene Artefakte
            \begin{enumerate}
                \item Präsentationen
                \item Dokumentation
            \end{enumerate}
        \item Visualisierung als GUI und/oder Commandline Interface von
            \begin{enumerate}
                \item abstrakter Syntax (AST) als ASCII, xml, SVG, \dots
                \item drei Adresscode als Text, Tripel-, Quadrupeldarstellung,
                    \dots
                \item anderen Ergebnissen: Interpreter, Debugger, \dots
            \end{enumerate}
    \end{itemize}
\end{frame}

\section{Einteilung in Gruppen}
\begin{frame}
    \frametitle{Aufteilung in Gruppen}
    \begin{block}{Gruppengröße}
        \begin{itemize}
            \item $\approx 25$ Teilnehmer im KVV
            \item 2 Gruppen mit $\approx 12,5$ Teilnehmer
        \end{itemize}
    \end{block}
    \begin{block}{Organisation}
        \begin{itemize}
            \item grundsätzlich frei gestellt
            \item jedoch legen wir Wert auf:
                \begin{enumerate}
                    \item Inkrementelle Software Entwicklung
                    \item Implementierung
                    \item Interface Spezifikation
                    \item Automatisierung der verschiedenen Tests
                    \item Visualisierung der Ergebnisse über Commandline, GUI,
                        \dots
                \end{enumerate}
        \end{itemize}
    \end{block}
\end{frame}

\section{Organisatorisches}
\subsection{Treffen}
\begin{frame}
  \frametitle{Projekttreffen}
  \begin{block}{Treffen aller Teilnehmer}
    \begin{itemize}
      \item alle zwei Wochen
      \item donnerstags von 14 bis 16 Uhr c.t.
      \item bei zu vielen Fehlterminen wird Anwesenheitspflicht eingeführt!
      \item von jedem Teilnehmer wird erwartet mindestens \emph{1x} zu
          präsentieren
    \end{itemize}
  \end{block}
  $\Rightarrow$ andere Woche für projektinterne Treffen vormerken!

  \begin{block}{Zweck der Projekttreffen}
    \begin{enumerate}
        \item Arbeitsfortschritt
        \item Projektmanagement
        \item Visualisierung der Ergebnisse
        \item Probleme, Fragen, Diskussion, \dots
    \end{enumerate}
  \end{block}
\end{frame}

\subsection{Bürozeiten der Betreuer}
\begin{frame}
    \frametitle{Bürozeiten der Betreuer}
    Betreuer des Softwareprojekts sind
    \begin{enumerate}
        \item Till Zoppke Email:~\email{zoppke@zedat.fu-berlin.de}
        \item Maximilian Konzack Email:~\email{maximilian.konzack@fu-berlin.de}
        \item Yves Müller Email:~\email{yves.mueller@fu-berlin.de}
    \end{enumerate}
    \begin{block}{Wo und wann?}
        Donnerstags 16 bis 18 Uhr im SR~158
    \end{block}
    \begin{block}{Wann zu nutzen?}
        bei
        \begin{itemize}
            \item Problemen im Team
            \item Unklarheiten
            \item Fragen
            \item Anregungen
            \item \dots
        \end{itemize}
    \end{block}
\end{frame}

\subsection{Zuteilung der Projekte}
\begin{frame}
  \frametitle{Zuteilung der Projekte}
  \begin{enumerate}
    \item Verteilen der Listen
    \item Maximal 2x Eintragen
    \item Wünsche bitte über Anmerkung notieren
    \item Zuteilung erfolgt heute!
  \end{enumerate}
\end{frame}

\section{Ende}
\begin{frame}
  \frametitle{Ende}
  \begin{center}
    \Huge
    Danke für die Aufmerksamkeit.\\[1em]
    \includegraphics[height=5cm]{compilers}
    \ 
  \end{center}
\end{frame}

\begin{frame}
  \frametitle{Weiteres Vorgehen}
  \begin{enumerate}
    \item Gruppeneinfindung
      \begin{itemize}
        \item interne Treffen organisieren
        \item Ansprechpartner für Tim und Max
        \item Erste Konzepte entwerfen: Interfaces, Beispielsyntax, \dots
        \item Probleme festhalten
        \item Absprache mit anderen Teams?
      \end{itemize}
    \item Repository einrichten
    \item Quellsprache verstehen
    \item LLVM näher kennen lernen
    \item Literatur konsultieren
  \end{enumerate}
\end{frame}

% All of the following is optional and typically not needed. 
\appendix
\section*{\appendixname}

\begin{frame}
  \frametitle{Literatur}
    
  \begin{thebibliography}{10}
    
  \beamertemplatebookbibitems
  % Start with overview books.

  
  \bibitem[AUS08]{Aho08}
    Alfred~V. Aho, Jeffrey Ullman, and Ravi Sethi.
    \newblock {\em Compiler: {P}rinzipien, {T}echniken und {W}erkzeuge}.
    \newblock Pearson Studium, 2. edition, 2008.


  \bibitem[Fin]{FindBugsWeb}
    {F}ind{B}ugs -- {F}ind {B}ugs in {J}ava {P}rograms.
    \newblock {\url{http://findbugs.sourceforge.net/}}.


  \bibitem[LA04]{Lattner04}
    Chris Lattner and Vikram~S. Adve.
    \newblock {LLVM}: {A} compilation framework for lifelong program analysis \&
      transformation.
      \newblock In {\em CGO}, pages 75--88. IEEE Computer Society, 2004.

    \bibitem[Lat02]{Lattner00diss}
      Chris~Arthur Lattner.
      \newblock {LLVM}: An infrastructure for multi-stage optimization, 2002.

    \bibitem[Sco09]{Scott09}
      Michael~Lee Scott.
      \newblock {\em Programming language pragmatics}.
      \newblock Morgan Kaufmann Publishers, 3. edition, 2009.
    
  \end{thebibliography}
\end{frame}

\end{document}
