\documentclass[ucs,9pt]{beamer}

% Copyright 2004 by Till Tantau <tantau@users.sourceforge.net>.
%
% In principle, this file can be redistributed and/or modified under
% the terms of the GNU Public License, version 2.
%
% However, this file is supposed to be a template to be modified
% for your own needs. For this reason, if you use this file as a
% template and not specifically distribute it as part of a another
% package/program, I grant the extra permission to freely copy and
% modify this file as you see fit and even to delete this copyright
% notice.
%
% Modified by Tobias G. Pfeiffer <tobias.pfeiffer@math.fu-berlin.de>
% to show usage of some features specific to the FU Berlin template.

% remove this line and the "ucs" option to the documentclass when your editor is not utf8-capable
\usepackage[utf8x]{inputenc}    % to make utf-8 input possible
\usepackage[english,german]{babel}     % hyphenation etc., alternatively use 'german' as parameter
\usepackage{hyperref}

\include{fu-beamer-template}  % THIS is the line that includes the FU template!

\usepackage{arev,t1enc} % looks nicer than the standard sans-serif font
% if you experience problems, comment out the line above and change
% the documentclass option "9pt" to "10pt"

% image to be shown on the title page (without file extension, should be pdf or png)
\titleimage{fu_500}

\newcommand{\email}[1]{\href{mailto:#1}{#1}}

\title{SWP Übersetzerbau im SS 13}
\subtitle{Einführung und Organisatorisches}
\author{Till~Zoppke \and Maximilian~Konzack \and Yves~Müller}
\institute[FU Berlin]{Freie Universität Berlin}

\date{Auftaktveranstaltung am 13.~April 2013}


% you can redefine the text shown in the footline. use a combination of
% \insertshortauthor, \insertshortinstitute, \insertshorttitle, \insertshortdate, ...
\renewcommand{\footlinetext}{\insertshortinstitute, \insertshorttitle, \insertshortdate}

% Delete this, if you do not want the table of contents to pop up at
% the beginning of each subsection:
%\AtBeginSubsection[]
%{
%  \begin{frame}<beamer>{Outline}
%    \tableofcontents[currentsection,currentsubsection]
%  \end{frame}
%}

\begin{document}

\begin{frame}[plain]
  \titlepage
\end{frame}

\begin{frame}{Übersicht}
  \tableofcontents
\end{frame}

\section{Projektidee}
\begin{frame}
  \frametitle{Entwicklung eines Compilers im Team}
  \begin{center}
    \includegraphics[width=0.7\textwidth]{pair_prog}
  \end{center}
\end{frame}
\begin{frame}
  \frametitle{Modularer Compiler}
  \begin{block}{Idee}
    \begin{itemize}
      \item Implementierung eines Übersetzers
      \item soll im Rahmen der Übersetzerbau genutzt werden können
      \item Quellsprache ist imperativ und statisch typisiert
    \end{itemize}
  \end{block}

  \begin{columns}
    \begin{column}{0.3\textwidth}
  \begin{center}
    \includegraphics[height=3cm]{java_duke}
  \end{center}
\end{column}
    \begin{column}{0.7\textwidth}
  \begin{block}{Ziele}
      \begin{description}
          \item[modular] Abgrenzung gegenüber anderen Modulen
          \item[einfach] Verwendbarkeit mit anderen Komponenten
          \item[getestet] Black- und Whiteboxtests
          \item[dokumentiert] im Quellcode und als Text
      \end{description}
  \end{block}
\end{column}
  \end{columns}
\end{frame}

\begin{frame}
    \frametitle{Veranschaulichung der zu erstellenden Artefakte}
    \begin{center}
        \includegraphics[width=0.9\textwidth]{pipeline}
    \end{center}
    \begin{itemize}
        \item bekannte Aufteilung in Front- und Backend
        \item Projektmanagement bezogene Artefakte
            \begin{enumerate}
                \item Präsentationen
                \item Dokumentation
            \end{enumerate}
        \item Visualisierung als GUI und/oder Commandline Interface von
            \begin{enumerate}
                \item abstrakter Syntax (AST) als ASCII, xml, SVG, \dots
                \item drei Adresscode als Text, Tripel-, Quadrupeldarstellung,
                    \dots
                \item anderen Ergebnissen: Interpreter, Debugger, \dots
            \end{enumerate}
    \end{itemize}
\end{frame}

\section{Einteilung in Gruppen}
\begin{frame}
    \frametitle{Aufteilung in Gruppen}
    \begin{block}{Gruppengröße}
        \begin{itemize}
            \item $\approx 25$ Teilnehmer im KVV
            \item 2 Gruppen mit $\approx 12,5$ Teilnehmer
        \end{itemize}
    \end{block}
    \begin{block}{Organisation}
        \begin{itemize}
            \item grundsätzlich frei gestellt
            \item jedoch legen wir Wert auf:
                \begin{enumerate}
                    \item Inkrementelle Software Entwicklung
                    \item Implementierung
                    \item Interface Spezifikation
                    \item Automatisierung der verschiedenen Tests
                    \item Visualisierung der Ergebnisse über Commandline, GUI,
                        \dots
                \end{enumerate}
        \end{itemize}
    \end{block}
\end{frame}

\section{Organisatorisches}
\subsection{Treffen}
\begin{frame}
  \frametitle{Projekttreffen}
  \begin{block}{Treffen aller Teilnehmer}
    \begin{itemize}
      \item alle zwei Wochen
      \item donnerstags von 14 bis 16 Uhr c.t.
      \item bei zu vielen Fehlterminen wird Anwesenheitspflicht eingeführt!
      \item von jedem Teilnehmer wird erwartet mindestens \emph{1x} zu
          präsentieren
    \end{itemize}
  \end{block}
  $\Rightarrow$ andere Woche für projektinterne Treffen vormerken!

  \begin{block}{Zweck der Projekttreffen}
    \begin{enumerate}
        \item Arbeitsfortschritt
        \item Projektmanagement
        \item Visualisierung der Ergebnisse
        \item Probleme, Fragen, Diskussion, \dots
    \end{enumerate}
  \end{block}
\end{frame}

\subsection{Bürozeiten der Betreuer}
\begin{frame}
    \frametitle{Bürozeiten der Betreuer}
    Betreuer des Softwareprojekts sind
    \begin{enumerate}
        \item Till Zoppke Email:~\email{zoppke@zedat.fu-berlin.de}
        \item Maximilian Konzack Email:~\email{maximilian.konzack@fu-berlin.de}
        \item Yves Müller Email:~\email{yves.mueller@fu-berlin.de}
    \end{enumerate}
    \begin{block}{Wo und wann?}
        Donnerstags 16 bis 18 Uhr im SR~158
    \end{block}
    \begin{block}{Wann zu nutzen?}
        bei
        \begin{itemize}
            \item Problemen im Team
            \item Unklarheiten
            \item Fragen
            \item Anregungen
            \item \dots
        \end{itemize}
    \end{block}
\end{frame}

\subsection{Bewertung}
\begin{frame}
    \frametitle{Bewertungsschema}
    umfasst
    \begin{enumerate}
        \item Quellcode
        \item Dokumentation
        \item Präsentationen
        \item Abschluss
    \end{enumerate} bezüglich der Meilensteine
    \begin{block}{Meilensteine}
        \begin{description}
            \item[M1] Arithmetik
            \item[M2] \texttt{print} Anweisung und Verzweigungen
            \item[M3] Schleifen und Arrays
        \end{description}
    \end{block}
\end{frame}

\subsection{Repositories}
\begin{frame}
    \frametitle{Verwaltung des SWPs auf GitHub}
    Link zur GitHub Organisation für das SWP:
    \url{https://github.com/organizations/swp-uebersetzerbau-ss13}
  \begin{columns}
    \begin{column}{0.5\textwidth}
  \begin{center}
    \includegraphics[height=2cm]{githuboctacat}
  \end{center}
\end{column}
    \begin{column}{0.5\textwidth}
  \begin{block}{Repositories}
      \begin{enumerate}
          \item Ein \emph{allgemeines Repository} für projektübergreifende
              \begin{itemize}
                  \item Dokumentation
                  \item Beispiele
                  \item Tests
                  \item Interfaces
              \end{itemize}
          \item \emph{jede} Gruppe erhält eigenes Repository für ihre
              Implementierung
      \end{enumerate}
  \end{block}
\end{column}
  \end{columns}
\end{frame}

\section{Git Primer}
\begin{frame}
    \frametitle{Einführung in git}
  \begin{columns}
    \begin{column}{0.3\textwidth}
  \begin{center}
    \includegraphics[width=3cm]{git_logo}
  \end{center}
\end{column}
    \begin{column}{0.7\textwidth}
  \begin{block}{Was ist git?}
      \begin{enumerate}
          \item Versionsverwaltung von Dateien, insbesondere für Quellcode
          \item freie Software
          \item geeignet für kleine bis große Projekte
          \item kann nur nur lokal (auf einem Rechner) oder stark verteilt
              genutzt werden
          \item viele große Open Source Projekte nutzen git
      \end{enumerate}
  \end{block}
\end{column}
  \end{columns}
  \end{frame}

  \begin{frame}[fragile]
    \frametitle{Wichtige Befehle für die Arbeit mit git}
    \begin{enumerate}
        \item Lokale Kopie vom Repository anlegen:
            \begin{verbatim}
            $ git clone <repo>
            \end{verbatim}
        \item Revision ansehen:
            \begin{verbatim}
            $ git show <rev number>
            \end{verbatim}
        \item Lokalen Veränderungen (noch ohne commit) ansehen:
            \begin{verbatim}
            $ git status
            \end{verbatim}
        \item Commit auf lokalem Repo:
            \begin{verbatim}
            $ git commit -m "message" -a|<file>|<dir>
            \end{verbatim}
        \item Veränderungen vom remote Repo ziehen:
            \begin{verbatim}
            $ git pull origin <branch>
            \end{verbatim}
        \item Eigene Commits auf remote Repo hoch laden:
            \begin{verbatim}
            $ git push origin <branch>
            \end{verbatim}
    \end{enumerate}
\end{frame}

\begin{frame}[fragile]
    \frametitle{Befehle zur Verwaltung von Branches}
    \begin{enumerate}
        \item Erstelle einen neuen Branch (und wechsle zu ihm):
            \begin{verbatim}
            $ git checkout -b <branch>
            \end{verbatim}
        \item Zeige alle verfügbaren Branches:
            \begin{verbatim}
            $ git branch -a
            \end{verbatim}
        \item Wechsele vom aktuellen Branch zu angegeben:
            \begin{verbatim}
            $ git checkout <branch>
            \end{verbatim}
        \item Merge von aktuellen Branch mit angegeben:
            \begin{verbatim}
            $ git merge <branch>
            \end{verbatim}
    \end{enumerate}
\end{frame}

\section{Ende}
\begin{frame}
  \frametitle{Ende}
  \begin{center}
    \Huge
    Danke für die Aufmerksamkeit.\\[1em]
    \includegraphics[height=5cm]{compilers}
    \ 
  \end{center}
\end{frame}

% All of the following is optional and typically not needed. 
\appendix
\section*{\appendixname}

\begin{frame}
  \frametitle{Literatur}
    
  \begin{thebibliography}{10}
    
  \beamertemplatebookbibitems
  % Start with overview books.

  
  \bibitem[AUS08]{Aho08}
    Alfred~V. Aho, Jeffrey Ullman, and Ravi Sethi.
    \newblock {\em Compiler: {P}rinzipien, {T}echniken und {W}erkzeuge}.
    \newblock Pearson Studium, 2. edition, 2008.


  \bibitem[Fin]{FindBugsWeb}
    {F}ind{B}ugs -- {F}ind {B}ugs in {J}ava {P}rograms.
    \newblock {\url{http://findbugs.sourceforge.net/}}.
  
   \bibitem[GitHub]{GitHubWeb}
    {G}it{H}ub
    \newblock {\url{https://github.com/}}.

   \bibitem[GitRef]{GitRefWeb}
       {G}it {R}eference
    \newblock {\url{http://gitref.org/}}.

   \bibitem[GitNotes]{GitNotesWeb}
       {S}ome {N}otes on {G}it
    \newblock {\url{http://java.dzone.com/articles/some-notes-git}}.

    \bibitem[Sco09]{Scott09}
      Michael~Lee Scott.
      \newblock {\em Programming language pragmatics}.
      \newblock Morgan Kaufmann Publishers, 3. edition, 2009.
    
  \end{thebibliography}
\end{frame}

\end{document}
